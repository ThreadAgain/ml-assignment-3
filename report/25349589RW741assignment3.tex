\documentclass[conference, 10pt]{IEEEtran}
% \IEEEoverridecommandlockouts
% The preceding line is only needed to identify funding in the first footnote. If that is unneeded, please comment it out.
\usepackage{cite}
\usepackage{amsmath,amssymb,amsfonts}
\usepackage{algorithmic}
\usepackage{graphicx}
\usepackage{textcomp}
\usepackage{xcolor}
\usepackage{float}
\begin{document}

\title{Time Series Forecasting using Recurrent Neural Networks}

\author{\IEEEauthorblockN{Schalk Visagie}
\IEEEauthorblockA{\textit{Computer Science Department} \\
\textit{Stellenbosch University}\\
Stellenbosch, South Africa \\
25349589@sun.ac.za}
}

\maketitle

\begin{abstract}

\end{abstract}

\begin{IEEEkeywords}

\end{IEEEkeywords}

\section{Introduction}



\section{Background}

\textbf{Elman Recurrent Neural Network }

The Elman recurrent neural network (Elman-RNN) is a simple RNN architecture consisting of an input layer, a hidden
layer, a context layer, and an output layer. The context layer, with the same number of neurons as the hidden layer,
stores a copy of the previous hidden state and feeds it back to the hidden layer, enabling the network to maintain
short-term memory of past inputs. This feedback loop introduces temporal dynamics, distinguishing it from feedforward
networks.

Elman-RNN is suitable for time-series prediction because its recurrent structure captures sequential dependencies,
allowing it to model patterns in data where current values depend on historical context, such as a simple sequence of
XOR operations, stock prices or weather.

TODO: Mathematically, the hidden state is formulated as 
% \( h_j(t) = F\left(\sum_{i=1}^m w_{ij} x_i(t) + \sum_{l=1}^n w_{lj}'
% h_l(t-1)\right) \), where \( F \) is the activation function (e.g., sigmoid), \( w_{ij} \) are input-to-hidden weights,
% and \( w_{lj}' \) are context-to-hidden weights. The output is \( y(t) = F\left(\sum_{j=1}^n w_j h_j(t)\right) \), with
% \( w_j \) as hidden-to-output weights.

% $h_t = \tanh(x_t W_{ih}^T + b_{ih} + h_{t-1}W_{hh}^T + b_{hh})$

\textbf{Jordan Recurrent Neural Network }

The Jordan recurrent neural network (Jordan-RNN) features an input layer, hidden layer, context layer, and output layer. Unlike
Elman-RNN, the context layer stores previous output values and feeds them back to the hidden layer, creating a feedback
loop from outputs to influence future computations. The Jordan-RNN's context layer has the same amount of
neurons as its output layer in contrast to the Elman-RNN having a context layer with the same size as its hidden layer.

Jordan-RNN applies to time-series prediction by leveraging output feedback to handle temporal correlations, making it effective
for forecasting tasks like price trends where past predictions inform future ones.

TODO:The mathematical formulation for the hidden state is 
% \( h_j(t) = F\left(\sum_{i=1}^m w_{ij} x_i(t) + \sum_{l=1}^n
% w_{lj}' y(t-l)\right) \), with \( y(t-l) \) from prior outputs. The output state is \( y(t) = F\left(\sum_{j=1}^n w_j
% h_j(t)\right) \).

\textbf{Multi-Recurrent Neural Network }

The multi-recurrent neural network (MRNN) integrates features of both Elman-RNN and Jordan-RNN. It includes input,
hidden, context, and output layers. The context layer receives feedback from both previous hidden states and
outputs and presented to the hidden layer at each time step. This hybrid architecture enhances memory by combining multiple
recurrent paths.

MRNN is apt for time-series prediction as its dual feedback mechanisms provide an alternative way to capture complex
temporal dependencies in sequential data compared to the previous two single-loop variants.

While specific equations vary, MRNN extends Elman-RNN and Jordan-RNN formulations by incorporating both hidden and output feedbacks
in the hidden state computation, such as combining terms from \( h(t-1) \) and \( y(t-1) \) in the activation.

\textbf{Backpropagation Training Process} 

The training process relies on backpropagation through time (BPTT) to effectively manage the sequential nature of the
data. This method works by unfolding the recurrent network across multiple time steps, essentially converting it into a
layered feedforward network where the weights are shared among the time-unfolded layers, allowing the use of standard
backpropagation techniques to calculate gradients \cite{quarkmlBackpropagationThrough}.

The process starts with a forward pass through the network. For each time step $t$ in the sequence, the hidden state is
calculated as \( h_t = \tanh(W_{xh} \cdot X_t + W_{hh} \cdot h_{t-1} + b_h) \), where \( X_t \) represents the input at
that step, \( W_{xh} \) is the weight matrix from input to hidden, \( W_{hh} \) is the recurrent weight matrix from
previous hidden to current hidden, and \( b_h \) is the hidden bias. Following this, the output at each time step is
computed as \( y_t = \phi(W_{oh} \cdot h_t + b_o) \), with \( W_{oh} \) as the weight matrix from hidden to output, \(
b_o \) as the output bias, tanh as the activation for the hidden layer, and \( \phi \) (such as sigmoid or softmax) for
the output layer \cite{quarkmlBackpropagationThrough}.

Next, the loss is evaluated, often using mean squared error for regression tasks like time-series prediction: \( L =
\frac{1}{N} \sum_{t=1}^{N} (y - y_t)^2 \), where y is the target value and N is the number of time steps. The derivative
of this loss with respect to the predicted output is \( \frac{\partial{L}}{\partial{y_t}} = \frac{2}{n}\sum_{t=t}^{n}(y
- y_t) \) providing the starting point for error propagation.

The backward pass then proceeds by applying the chain rule to propagate errors back through the unfolded network,
accumulating gradients over all time steps. For the output weights, the gradient is \( \frac{\partial L}{\partial
W_{oh}} = \sum_{i=0}^{t} \frac{\partial L}{\partial y_{t-i}} \cdot \frac{\partial y_{t-i}}{\partial W_{oh}} \), which
sums contributions from each relevant time step. For the input-to-hidden weights, it is \( \frac{\partial L}{\partial
W_{xh}} = \sum_{i=0}^{t} \left[ \left( \frac{\partial L}{\partial y_{t-i}} \cdot \frac{\partial y_{t-i}}{\partial
h_{t-i}} \right) \cdot \left( \prod_{j=(t-i+1)}^{t} \frac{\partial h_j}{\partial h_{j-1}} \right) \cdot \frac{\partial
h_{t-i}}{\partial W_{xh}} \right] \), accounting for how errors flow through the recurrent connections. Similarly, for
the hidden-to-hidden weights: \( \frac{\partial L}{\partial W_{hh}} = \sum_{i=0}^{t} \left[ \left( \frac{\partial
L}{\partial y_{t-i}} \cdot \frac{\partial y_{t-i}}{\partial h_{t-i}} \right) \cdot \left( \prod_{j=(t-i+1)}^{t}
\frac{\partial h_j}{\partial h_{j-1}} \right) \cdot \frac{\partial h_{t-i}}{\partial W_{hh}} \right] \). 

Once these gradients are computed, the weights are updated using gradient descent with a learning rate \( \alpha \).
Care must be taken with long sequences to mitigate issues like vanishing or exploding gradients that can arise from
repeated multiplications in the product terms \cite{quarkmlBackpropagationThrough}.

In the context of specific architectures, BPTT in Elman RNNs emphasizes updating the hidden-to-context feedback to
capture internal dynamics; in Jordan RNNs, it prioritizes the output-to-context loop for incorporating prior
predictions; and in multi-RNNs, it handles both types of feedback simultaneously to support more robust temporal
modeling in time-series applications.

\section{Methodology}
\subsection{Datasets}
TODO: ensure Data Preprocessing justifications

\textbf{S\&P 500 ETF Daily OHLCV} dataset was obtained from Yahoo Finance. This dataset comprises 5,031 trading days of
Open, High, Low, Close, and Volume data from 19 September 2005 to 19 September 2025. This dataset is particularly
relevant for evaluating RNN architectures due to its inherent non-stationarity, volatility shifts, and complex
non-linear dependencies, which provide a robust benchmark for comparing the ability of different models to learn
temporal patterns and predict next-step log returns. Preprocessing was necessary to ensure data quality and model
stability. The 85 instances identified as outliers ($\sigma >$ 3) were clamp-transformed to 3 standard deviations from
the mean. To address multicollinearity among the highly correlated OHLC features while retaining discriminatory power,
the Open, High, and Low variables were dropped and replaced with two engineered features: the high-low range and log
returns. The closing price, confirmed as non-stationary by Augmented Dickey-Fuller (ADF) and
Kwiatkowski-Phillips-Schmidt-Shin (KPSS) tests, was stabilized by conversion to log returns. Finally, all features were
scaled to a [0, 1] range independently for the training and testing sets to prevent data leakage.

\textbf{VIX Daily OHLC} dataset, sourced from Yahoo Finance, provides 5,031 observations of daily Open, High, Low, and
Close data for the period spanning 19 September 2005 to 19 September 2025. This time series is well-suited for testing
RNNs because its structural properties. These structural properties include sharp fluctuations, sudden spikes, and
shifting sequential dependencies that create challenging non-stationary signals. These characteristics differ
significantly from the other datasets which make it an ideal candidate for diversifying training data. Data preparation
involved several steps. First, 73 outliers exceeding 3 standard deviations were clamped. The non-stationarity of the
closing price, verified with ADF and KPSS tests, was resolved by transforming the series into log returns. Similar to
the S\&P 500 data. The redundant and highly correlated Open, High, and Low features were removed and replaced with the
engineered high-low range and log return features. The resulting feature set was then scaled to a [0, 1] range
separately for the training and testing partitions.

The \textbf{ElectricityLoadDiagrams20112014} dataset contains high-frequency electricity consumption readings in
kilowatts for 370 clients, recorded every 15 minutes from January 2011 to the end of 2014. For this assignment, a single
client's consumption profile, consisting of 140,256 measurements, was arbitrarily selected to create a univariate
time-series forecasting scenario. This provides a distinct high-granularity, cyclical test case for evaluating the
predictive performance of the RNN models. The data required minimal cleaning as it contained no missing values. However,
power consumption values exceeding 3 standard deviations from the mean were clamp-transformed to handle outliers. Both
ADF and KPSS tests confirmed that the series was non-stationary. To address this, a 24-hour seasonal differencing was
applied, accounting for the inherent daily cyclicality of energy usage. During model training and evaluation, the data
was scaled to a [0, 1] range independently for the training and testing sets.

The \textbf{Synthetic Autoregressive Stationary (AR(1))} dataset was generated using Python code with the NumPy library,
simulating a univariate time series from an AR(1) process defined by the equation $x_t = 0.5 x_{t-1} + \epsilon_t$,
where $\epsilon_t$ is white noise drawn from a normal distribution with mean 0 and standard deviation 1. This dataset
consists of 10,000 sequential observations, indexed from time step 0 to 9,999, following a 500-point burn-in period to
ensure the process reaches stationarity. It is particularly suitable for benchmarking RNN architectures in this project
due to its inherent stationarity, linear autoregressive dependencies, and absence of trends or seasonality, offering a
controlled environment to evaluate the models' ability to capture simple recurrent patterns without the confounding
factors present in real-world data, thereby serving as a baseline for comparison with non-stationary datasets. The
synthetic nature ensured no missing values or structural anomalies. Stationary was confirmed both by design
(autoregressive coefficient $|\phi| = 0.5 < 1$) and through ADF and KPSS tests, requiring no differencing or detrending.
Finally, the series was scaled to a [0, 1] range independently for the training and testing sets to facilitate stable
RNN training and prevent data leakage.

The \textbf{TimeSeries Weather Dataset}, sourced from Kaggle, contains hourly historical weather data for two locations.
The location having more records (389,496 observations spanning January 1, 1980, to June 6, 2024) was selected. This
dataset is a multivariate time series with 17 continuous features, including temperature, humidity, dew point,
precipitation, pressure and cloud cover to name a few. This dataset was chosen for its high temporal granularity,
pronounced daily and seasonal cycles as well as multivariate interactions. Its inclusion enhances dataset diversity by
introducing time-series data with multiple cyclical tendencies and non-stationarity, contrasting the previously
mentioned datasets. The data exhibited no missing values or major irregularities. Outliers exceeding 3 standard
deviations were clamp-transformed. Highly correlated features with absolute correlation greater than 0.75 were dropped
to reduce multicollinearity while preserving predictive power. The target feature is defined as the next hour's
differenced temperature. Non-stationarity, confirmed by ADF and KPSS tests, was addressed through 24-hour seasonal
differencing to account for daily periodicity. All features were scaled to a [0, 1] range independently for the training
and testing sets to ensure model stability and prevent data leakage.

\subsection{Recurrent Neural Networks Implemented}
TODO: describe the optimisation algorithm used
TODO: describe the loss function used
\subsection{Cross Validation Implementation}

\subsection{Overfitting/Underfitting Prevention}


\section{Empirical Procedure}


\section{Results}

\section{Conclusion}
\cite{yahoosnp500}
\cite{yahoovix}
\cite{elec_cons}

\bibliographystyle{IEEEtran}
\nocite{myrepo}
\bibliography{refs.bib}


\end{document}